%\documentclass{article}
\documentclass[manuscript]{acmart}
%\documentclass[format=acmsmall, screen=true, review=false]{acmart}
%\documentclass{acmart}

\usepackage{caption}
\usepackage{subcaption}
\usepackage{graphicx}
\usepackage{enumitem}


\newcommand{\lb}[1]{\textcolor{orange}{LB: #1}}
\newcommand{\dname}{DHRD}
\newcommand{\ie}{i.e.,~}
\newcommand{\eg}{e.g.,~}
\newcommand{\datalink}{https://github.com/deliveryhero/dh-reco-dataset}

\graphicspath{{figs/}}


\begin{document}


\title{Delivery Hero Recommendation Dataset: A Novel Dataset for Benchmarking Recommendation Algorithms}

\author{Yernat Assylbekov$^*$, Raghav Bali$^*$, Luke Bovard$^*$, Christian Klaue$^*$}
\affiliation{
\institution{Delivery Hero}
\city{Berlin}
\country{Germany}}
\email{yernat.assylbekov@deliveryhero.com, raghav.bali@deliveryhero.com,
luke.bovard@deliveryhero.com, christian.klaue@deliveryhero.com}
\def\thefootnote{*}\footnotetext{These authors contributed equally to this work}\def\thefootnote{\arabic{footnote}}
%\author{Yernat Assylbekov}
%\affiliation{%
%\institution{Delivery Hero}
%\city{Berlin}
%\country{Germany}}
%\email{yernat.assylbekov@deliveryhero.com}
%\author{Raghav Bali}
%\affiliation{%
%\institution{Delivery Hero}
%\city{Berlin}
%\country{Germany}}
%\email{raghav.bali@deliveryhero.com}
%\author{Luke Bovard}
%\orcid{0000-0003-3555-062X}
%\affiliation{%
%\institution{Delivery Hero}
%\city{Berlin}
%\country{Germany}}
%\email{luke.bovard@deliveryhero.com}
%\author{Christian Klaue}
%\affiliation{%
%\institution{Delivery Hero}
%\city{Berlin}
%\country{Germany}}
%\email{christian.klaue@deliveryhero.com}

\begin{abstract}
In this paper we propose Delivery Hero Recommendation
Dataset (\dname), a novel real-world dataset for researchers.
\dname\ comprises over a million food delivery orders from three distinct
cities, encompassing thousands of vendors and an extensive range of
dishes, serving a combined customer base of over a million individuals.
We discuss the challenges associated with such real-world datasets. By
releasing \dname, researchers are empowered with a valuable resource for
building and evaluating recommender systems, paving the way for
advancements in this domain.
\end{abstract}

\maketitle

%%%%%%%%%%%%%%%%%%%%%%%%%%%%%%%%%%%%%%%%%%%%%%%%%%
\section{Introduction}
\label{sec:intro}
%%%%%%%%%%%%%%%%%%%%%%%%%%%%%%%%%%%%%%%%%%%%%%%%%%
With the importance of machine learning in industrial applications,
coupled with the correlated increase of research publications and the
success of deep learning techniques in fields like NLP and computer
vision, one would expect that revolutionary progress resulting from these
publications would be reflected in the field of Recommendation Systems
(RS). While latest state-of-the-art approaches claim to outperform
traditional methods, further investigations reveal that this is not the
case, and fine-tuned simple ideas still produce comparable
results \cite{muller18a}.  

The reasons for these findings point to a number of issues including
poor reproducibility, low sample size, specific preprocessing techniques which
do not generalise. Differences in training, validation
and test split as well as removal of samples can be observed in papers
using the popular MovieLens dataset (\eg  \cite{sun2019} vs
 \cite{ludewig2018}). Enabling researchers and practitioners to
easily compare and reproduce results across works by providing them with
good quality evaluation datasets is one step in the correct direction to
enhance research.

A dataset which is used to evaluate different approaches should have the
following attributes:
\begin{enumerate}[topsep=0pt]
\item \textbf{Accessibility}: The dataset should be open-sourced, easily
accessible and well documented.
\item \textbf{Relevance}: The dataset should contain relevant features that are
important for the analysis and modelling task at hand.
\item \textbf{Quality}: The data should be of high quality, with minimal missing
values, errors, or inconsistencies.
\item \textbf{Reproducibility}: The training and test data should be clearly
defined to ensure comparisons. 
\item \textbf{Representativeness}: While maintaining relevance and quality, the data 
should also reflect challenges faced in real-world settings (eg: inconsistent labels, 
changing identifiers, size/quantity information mixed with product names, etc.)
\item \textbf{Multimodality}: The dataset should consist of different data types
like text or images to maximise the horizon of potential applicable
approaches.
\item \textbf{Volume}: The dataset should be large enough to provide a
representative sample of the population or phenomena being studied.
\end{enumerate}
In order to enable researchers to correctly evaluate and compare
approaches, it is fundamental to provide them with real world data
samples. Given those samples and resulting research success, the outcome
can then be used in the industry resulting in a fruitful cooperation.
Therefore, it is the responsibility of companies to produce this
real-world data and make it accessible to researchers.  For a large
proportion of the industries there exist datasets which are
widely used and considered representative. A few examples of such areas
include music (\eg Spotfiy \cite{spotify_dataset},
lastfm \cite{Bertin-Mahieux2011}), movies (\eg
MovieLens \cite{harper2015movielens}, IMDb\cite{imdb_dataset},
Netflix \cite{netflix}) and retail (\eg Amazon\cite{ni2019},
Tmall \cite{tmall_dataset}). 

In contrast, there are few datasets that relate to food recommendation
and this domain offers many interesting challenges that have been
under-represented in literature and, to the best of our knowledge, none
of the leading food delivery companies across the world have officially
released datasets for research purposes. There are datasets available
that have a subset of food related data, but they either are a very small
percentage of the whole dataset \cite{ecommerce} or represent only a
specific geographical area like US \cite{yelp} or have been
scraped \cite{zomato,grubhub} with unspecified usage and license agreements. 

Food recommendation has a variety of topics that make it a fruitful field
of research and investigation, including, but not limited to delivery
time, quality and fee, availability, nutritional breakdown, cuisine type,
group orders, dietary considerations, as well as cultural, social and
environmental factors.  Recommendations within these parameters are also
complex since dishes can be very different depending on the restaurants
or a customer undergoes a dietary change and low-quality recommendations
have long-term consequences.  Additionally, food items are often time
dependent, sometimes even ordered together and therefore influence each
other.



%%%%%%%%%%%%%%%%%%%%%%%%%%%%%%%%%%%%%%%%%%%%%%%%%%
\section{Dataset Description}
\label{sec:dataset}
%%%%%%%%%%%%%%%%%%%%%%%%%%%%%%%%%%%%%%%%%%%%%%%%%%
To address the concerns raised in the previous section, we have
constructed a dataset from Delivery Hero’s largest brands such as
\href{https://www.foodpanda.com/about-foodpanda/}{foodpanda} and
\href{https://www.foodora.com/about/}{foodora}.
Delivery Hero brands process more than 5 million orders daily
with operations in more than 400 cities \cite{foodpanda_info}. This
dataset, which encompasses millions of real world orders, will help
researchers better evaluate recommendation algorithms in the food
recommendation sector. 

The data we are releasing here\footnote{\datalink} covers a time period
of three months with data consisting of 4.5 million orders processed on
Delivery Hero platforms for three cities: Singapore, Taipei, and
Stockholm. The data consists of three parts for each location: orders,
products, and vendor information.  Orders data contains information about
orders being placed, product data contains data about the products
offered on a vendor's menu, and vendor data contains metadata about the
vendor. Table~\ref{table:schema} contains an overview of the columns in
the dataset and table~\ref{table:stats} presents summary statistics of
the dataset.  To ensure privacy and GDPR compliance, we have removed all
personally identifiable information and the data has been anonymized
through hashing with salt. Additionally, this data may not reflect the
present availability of products and vendors in those cities. The
normalised prices listed are based on those that the user would see and
contain no proprietary discount, incentives, or commissions related
information as those are beyond the scope of this data release.  Due to
privacy and business restrictions, we are not able to provide user
demographic or vendor popularity information in this data release. 

We have used a geohash \cite{geohash} to approximate user/vendor
locations. Using vendors that are within the same geohash / neighbouring
geohashes is an effective way to determine all available vendors for a
given user without real-time data.  Internally, recommendation models use
similar approximations and we have found it to give good results.

In contrast to the majority of open source datasets that are
pre-processed, clean, and carefully curated to eliminate real-world
challenges, this dataset emulates real-world settings. Our dataset comprises
missing values, inconsistent IDs, and other typical issues encountered in
real-world systems. The main motivation behind keeping the dataset as raw
as possible were to enable the community to work towards novel
pre-processing techniques, experience real-world issues, develop novel
ranking and recommendation methods and also ensure that we keep any known
biases while releasing such a dataset to the minimum. We list below a few
potential challenges that we encourage researchers to explore. 

\begin{enumerate}
\item \textbf{Non-Trivial Cuisines}: Although \eg Singapore offers an
extensive variety of 78 cuisines, a detailed examination of these
cuisines indicates a very interesting scenario. For instance, rice,
noodles, tea, and sandwiches are some of the top primary cuisines listed
in the dataset. Even though these are marked as primary cuisines,
intuitively these do not correspond to typical understanding of cuisines.
Another typical case is similar cuisines which cannot be
directly treated as one. For example, bubble tea, coffee, and beverages
could be grouped together as a single primary cuisines (say beverages),
whereas identifying a common category for sandwiches and burgers could
prove challenging. 

\item \textbf{Product Names}: Product/Dish/Beverage names play an important role
in understanding user preferences and tastes. Vendors also benefit from
understanding what products are in demand and which ones aren’t to manage
inventories, incentives and more. Vendors also use creative names,
marketing labels, latest trending terms, and add quantity/size qualifiers
to their offerings to make them more attractive to customers. This
creates unique challenges to utilise this information for understanding
ordering patterns, preferences and  presents an unique opportunity for
developing pre-processing techniques using NLP and domain understanding.

\item \textbf{Unordered Products}: Another unique challenge is the difference
between the available list of products in the system versus products
being ordered. For instance, in Taipei there are details for 810K
unique products out of which only 327K have associated orders (a mere
40\% of the total). This situation frequently occurs due to
a number of factors involving changes to product names (spelling
corrections, marketing labels, etc.), portion/quantity changes, price
modification, seasonal menu changes and more. This requires meticulous
exploration of the dataset to come up with novel preprocessing techniques
for different downstream tasks (not limited to recommendations alone).
Multiple languages add an additional layer of complexity to such methods.
\end{enumerate}



\begin{table}
\begin{tabular}{|p{0.15\linewidth} | p{0.55\linewidth}| p{0.35\linewidth}}
Column & Description  & Dataset \\ \hline
customer\_id & uniquely hased customer\_id & Orders\\
order\_time & time order was placed & Orders \\
day\_of\_week & day of the week the order was placed, with 0 = Sunday,
 & Orders \\
order\_day & day from the start of the dataset when the order was placed
& Orders \\
geohash & hashed geographic location of where the order was placed given
to 5 digits of precision & Orders, Vendors\\
order\_id &  unique order\_id   & Orders\\
vendor\_id & unique hashed vendor\_id of where the order was placed &
Orders, Vendors, Product \\
chain\_id & unique hashed identifier of the chain that a vendor belongs
to. Not all vendors belong to chain and if null the vendor is not part of
a chain & Vendor\\
geohash & hashed geographic location of where the order was placed given
to 5 digits of precision & Orders, Vendor \\
primary\_cuisine & vendor specified cuisine & Vendor \\
product\_id & unique hashed product\_id of the product ordered & Orders,
Product\\
name & the product name as displayed on the menu  & Product\\
unit\_price & the normalised unit price of the item. & Product
\end{tabular}
\caption{Schema of the three different dataset tables.}
\label{table:schema}
\end{table}


\begin{table}
    \begin{tabular}{|cccccc|}
    City &  \# of Orders & \# of Customers &  \# of Vendors & \# of Products
    & \# of Products Ordered\\ \hline
    Singapore & 1.99M & 512K  &7411 & 1.06M & 256K \\
    Stockholm & 400K & 122K & 1148  &111K & 41K\\
    Taipei & 2.0M & 741K & 9506 & 810K & 327K
    \end{tabular}
    \caption{Summary statistics of the dataset for each of the three cities}
    \label{table:stats}
\end{table}

%%%%%%%%%%%%%%%%%%%%%%%%%%%%%%%%%%%%%%%%%%%%%%%%%%
\section{Conclusion}
\label{sec:conclusion}
%%%%%%%%%%%%%%%%%%%%%%%%%%%%%%%%%%%%%%%%%%%%%%%%%%
This paper presents the \dname, a novel online food delivery dataset,
which serves to expand and enhance the existing collection of public
datasets for recommendation tasks across diverse domains. The dataset
offers authentic information with minimal preprocessing to preserve
real-world characteristics and inspire innovative approaches for
preprocessing and recommendation techniques. It is conveniently available
in a user-friendly format\footnote{\datalink}, accompanied by
comprehensive details and descriptions outlined in this paper.  While we
anticipate that this dataset will introduce unique challenges to the
research community, our objective is to continually enhance and enrich it
through subsequent versions by incorporating additional features,
modalities, and volume.

\section{Speaker Bio and Acknowledgements}
\label{sec:speaker}
The authors of this paper are a team of experienced Data Scientists who
have been actively engaged in data science and related activities at
Delivery Hero. Their primary focus has been on enhancing customer
recommendations across diverse markets. Yernat and Luke, possessing
doctorate degrees in Mathematics and AstroPhysics respectively,
successfully completed their doctoral studies at the University of
Washington and Goethe University Frankfurt. Raghav Bali, holding a
Masters from IIIT-Bangalore, brings to the team over a decade of valuable
experience gained from working with major technology companies. Christian
Klaue, equipped with a Masters from the University of St. Thomas, boasts
a versatile professional background spanning various domains. The authors
extend their heartfelt gratitude to Ankur Kaul, Koray Kayir, and the
entire squad for their unwavering support and invaluable inputs
throughout this work. Their support and guidance have been instrumental
in shaping the outcomes of this study.
%The authors of this paper have been working at Delivery Hero as Data Scientists
%to develop models for improving recommendations for our customers across markets.
%Yernat and Luke completed their PhDs from University of Washington and Goethe University Frankfurt in Mathematics and AstroPhysics respectively.
%Raghav Bali has a Masters from IIIT-Bangalore and has over 10 years of experience working with major technology companies.
%Christian Klaue also has a Masters from University of St. Thomas and has worked in different domains throughout. The authors are working towards building
%tailored solutions to improve overall user experience and business metrics. The authors would also like to thank Ankur Kaul, Koray Kayir and rest of the squad 
%for their support and inputs throughout.

\bibliographystyle{plain} 
\bibliography{refs} % Entries are in the refs.bib file
\end{document}
